% !TEX program = xelatex

\documentclass{resume}
\usepackage{zh_CN-Adobefonts_external} % Simplified Chinese Support using external fonts (./fonts/zh_CN-Adobe/)
%\usepackage{zh_CN-Adobefonts_internal} % Simplified Chinese Support using system fonts

\begin{document}
\pagenumbering{gobble} % suppress displaying page number

\name{李 军}

\basicInfo{
  \email{lijun215@mails.ucas.ac.cn} \textperiodcentered\ 
  \phone{(+86) 155-0126-3587} \textperiodcentered\ 
  \github[https://github.com/muzi-8]{https://github.com/muzi-8}}

\section{教育背景}
\datedsubsection{\textbf{中国科学院大学} \quad 中国科学院空天信息研究院地理与赛博空间信息技术研究室  \quad 硕博连读}{2015 -- 至今}
\datedsubsection{\textbf{西北农林科技大学}\quad 机械与电子工程学院 \quad 工学学士}{2011 -- 2015}
GPA: 3.8 / 4 \quad 专业排名:Top 1 / 64 \quad 保研

\section{项目经历}
\datedsubsection{\textbf{数字地球国产化} }{2016.9 -- 2018.12}
\role{项目描述}{实验室战略先导专项,利用遥感卫星这只“眼睛”来观察地球。该项目目标使用户能够对国产陆地卫星数据感觉好用、会用、用好。}
%Brief introduction: xxx.
\begin{itemize}
  %\item 项目描述:中科院战略先导专项,“面向感知中国的新一代信息技术研究”,目标是利用网络边缘设备,提供现场、弹性、自治服务。
  \item 静态目标插件开发
  \item 动态目标插件开发
  \item 目标可视化原型设计
  \item 国产化的平台适配
  \item Mysql数据库库表设计与数据录入
\end{itemize}

\datedsubsection{\textbf{智能信息可视分析平台} }{2017.3 -- 2017.9}
\role{项目描述}{实验室创新专项,利用B/S架构,进行大数据的分析与异常检测,异常事件的可视分析与预警}
%Brief introduction: xxx.
\begin{itemize}
	%\item 项目描述:中科院战略先导专项,“面向感知中国的新一代信息技术研究”,目标是利用网络边缘设备,提供现场、弹性、自治服务。
	\item 事件动态演变可视化
	\item 人物画像分析
	\item web技术进行信息可视化与可视分析
\end{itemize}

\datedsubsection{\textbf{光学遥感图像分类模型的可解释性}}{2018.9--至今 }
\role{项目描述}{实验室预言专项。由于深度学习模型的非线性以及参数的复杂性,使得深度学习模型作为一种黑盒应用于军事目标的识别与检测,严重造成了结果的不确定性与用户的不可信任性。目标是通过视觉与语义两个维度进行模型的可解释性,实现深度学习算法的透明化,增强用户对模型的信任度以及模型做出决策的合理性。}
%Brief introduction: xxx
\begin{itemize}
  \item 以可视分析为途径,对模型的结构、决策行为、模型特征学习等多维度进行深度学习的可解释性研究;
  \item \textbf{该课题为本人博士论文研究方向之一,研究仍在进行中}。
\end{itemize}

\datedsubsection{\textbf{大学生科创:新型PM2.5智能检测仪}}{2013.5 -- 2015.5}
\role{项目描述}{大学生科技创新项目,应对恶劣的雾霾天气环境对人们日常生活的影响,基于51单片机做出一套新型pm2.5智能检测仪,要求能够实现颗粒浓度检测,高危预警,可视显示,人机交互}
%Brief introduction: xxx
\begin{itemize}
	\item 负责项目可行性方案的确定;
	\item 独立编写单片机控制代码;
	\item 辅助PCB硬件设计;
	\item 负责整个项目的答辩
\end{itemize}

% Reference Test
%\datedsubsection{\textbf{Paper Title\cite{zaharia2012resilient}}}{May. 2015}
%An xxx optimized for xxx\cite{verma2015large}
%\begin{itemize}
%  \item main contribution
%\end{itemize}

\section{科研成果}
\begin{itemize}[parsep=0.5ex]
  \item \textbf{Li, J.}, Lin, D., Wang, Y., Xu, G., \& Ding, C. (2019). Deep Discriminative Representation Learning with Attention Map for Scene Classification. \textit{International Society for Photogrammetry and Remote Sensing, ISPRS} (SCI期刊在投,\textbf{TOP期刊},第一作者)
  \item Lin, D., Wang, Y., Xu, G., \textbf{Li, J.}, \& Fu, K. (2018). Transform a Simple Sketch to a Chinese Painting by a Multiscale Deep Neural Network. \textit{Algorithms.} (EI期刊,引用频次:1)
  \item 基金项目申请书:《面向光学遥感图像分类的深度学习可解释性研究》. (2019) (自然科学基金-青年科学基金项目申请中,\textbf{主要负责人})
  \item 专著:“地理空间大数据分析”. (2018)  (负责:第五章节:地理空间大数据可视化-面向方法的可视分析技术)
  \item 专利:“一种地球重力场数据的三维可视化方法”.(2017)申请号:CN201710105916.7
  \item 网络文章:“访学风采|北大可视化暑期学校”. (2018)(原创性文章,发表于实验室公众号“赛博智能”)
  \item 王富春; \textbf{李军}; 张润浩; 任静; 宋怀波;(2015) 基于计算机视觉的苹果霉心病病变程度测量方法.\textit{农机化研究 Journal of Agricultural Mechanization Research.}(中文核心期刊,第二作者,引用频次:3, 下载频次:184)
  \item  张润浩; \textbf{李军}; 任静; 宋怀波;(2015) 基于高斯自适应拟合的苹果目标分割方法研究.\textit{农机化研究 Journal of Agricultural Mechanization Research.}(中文核心期刊,第二作者,引用频次:3,下载频次:124)
  \item 科创:“新型PM2.5智能检测仪仪器” (2013-2015) (项目负责人)
\end{itemize}

\section{荣誉奖励}
\datedline{优秀党员,    中国科学院大学}{2017}
\datedline{励志奖学金,  西北农林科技大学}{2013-2014}
\datedline{校级三好学生,西北农林科技大学}{2012-2013}
\datedline{国家奖学金,  西北农林科技大学}{2012-2013}


\section{比赛经历}
\begin{itemize}[parsep=0.5ex]
	\item 2018数据可视分析大赛 Chinavis (职务:负责人);
	\item 2017数据可视分析大赛 Chinavis (职务:负责人);
	\item 2014 TI杯陕西省电子设计竞赛 (选题:F;获奖:三等奖);
\end{itemize}

\section{专业技能}
\begin{itemize}[parsep=0.5ex]
	\item 通过CET-6,能够流畅进行英文阅读、写作和交流;
	\item 熟练掌握C, Python等编程语言以及Linux系统;
	\item 熟练掌握基本数据结构和算法,有良好的编程风格;
	\item 熟练掌握机器学习领域、深度学习领域基本算法以及PyTorch等深度学习工具。
\end{itemize}

\section{实践经历}
\begin{itemize}[parsep=0.5ex]
  \item \datedline{Google TPU 线下交流与分享小组(学习小组:组长)}{2019.4-至今}	
  \item \datedline{跟随理纯公益绿化植树团奔赴河北省张北县进行为期两天的植树活动}{2017.11}
  \item \datedline{中国科学院大学学生会体育部(成员)}{2015.9-2016.6}
  \item \datedline{西北农林科技大学大学生艺术团组织部(成员)}{2011.9-2012.6}
\end{itemize}

\section{个人评价}
\begin{itemize}[parsep=0.5ex]
  \item 责任心强,团队合作意识强,工作态度积极,喜欢结交志同道合朋友;
  \item 学习能力强,喜欢接受新鲜事物,乐于分享自己的感想
  \begin{itemize}[parsep=0.5ex]
  	\item[*] 自学斯坦福吴恩达CS229 课程内容 
  	\item[*] 自学斯坦福李飞飞CS231n课程内容
  	\item[*] 自学国立台湾大学李宏毅机器学习2017课程内容
  	\item[*] 自学清华大学严蔚敏数据结构课程内容(正在进行)
  \end{itemize}
  \item 具有一定运动技能:
  \begin{itemize}[parsep=0.5ex]
  	\item[*] 擅长篮球运动
  	\begin{itemize}[parsep=0.5ex]
  		\item[*] 2016年电子电气与通信工程学院院系比赛季军。
  		\item[*] 2015年机械与电子工程学院院系专业比赛冠军。
  		\item[*] 2011年机械与电子工程学院院系新生杯冠军。
    \end{itemize}
    \item[*] 热爱健身、跑步
    \begin{itemize}[parsep=0.5ex]
    	\item[*] 2015年至今坚持跑步四年。
    	\item[*] 2016年参加朝阳公园半程马拉松。
    	\item[*] \textbf{2016年作为创始人之一成立学术跑团:AcademyRunners}
    	\begin{itemize}[parsep=0.5ex]
    		\item[*] 人数:249。
    		\item[*] 跑团宗旨:跑步是为了更好的科研。
    	\end{itemize}
    \end{itemize}
    \item[*] 喜欢吉他、音乐、看书。
    \begin{itemize}[parsep=0.5ex]
    	\item[*] 2016年6月与不同的院所同学共同成立线上读书交流会--“科苑阅读”。
    \end{itemize}
  \end{itemize}
  \item 个人网站:\href{https://muzi-8.github.io/}{https://muzi-8.github.io/}
\end{itemize}

%% Reference
%\newpage
%\bibliographystyle{IEEETran}
%\bibliography{mycite}
\end{document}
